\section{Instructions: Language of the computer}
\subsection{Operations of the Computer Hardware}
\begin{frame}{Operations of the Computer Hardware}
\begin{figure}\caption{Arithmatic Instructions in MIPS}
\begin{center}
\includegraphics[width=\textwidth, height=0.2\textheight]{docs/images/operations-1}
\end{center}
\end{figure}
\end{frame}

\begin{frame}{Operations of the Computer Hardware (Cont'd)}
\begin{table}[H]
\begin{adjustbox}{width=\textwidth}
\begin{tabular}{|c|c|c|c|}
\hline
load word & \texttt{lw \$s1, 20(\$s2)} & \texttt{\$s1 = Memory[\$s2 + 20]} & Word from memory to register \\
\hline
store word & \texttt{sw \$s1, 20(\$s2)} & \texttt{Memory[\$s2 + 20] = \$s1} & Word from register to memory \\
\hline
load byte & \texttt{lb \$s1, 20(\$s2)} & \texttt{\$s1 = Memory[\$s2 + 20]} & Byte from memory to register \\
\hline
load byte unsigned & \texttt{lbu \$s1, 20(\$s2)} & \texttt{\$s1 = Memory[\$s2 + 20]} & Byte from memory to register \\
\hline
store byte & \texttt{sb \$s1, 20(\$s2)} & \texttt{Memory[\$s2 + 20] = \$s1} & Byte from register to memory \\
\hline
load upper immed & \texttt{lui \$s1, 20} & \texttt{\$s1 = 20 * $2^{16}$} & Loads constant in upper 16 bits \\
\hline
\end{tabular}
\end{adjustbox}
\caption{Data Transfer Instructions in MIPS}
\end{table}
\end{frame}

\begin{frame}{Operations of the Computer Hardware (Cont'd)}
\begin{figure}\caption{Logical Instructions in MIPS}
\begin{center}
\includegraphics[width=\textwidth, height=0.4\textheight]{docs/images/operations-3}
\end{center}
\end{figure}
\end{frame}

\begin{frame}{Operations of the Computer Hardware (Cont'd)}
\begin{figure}\caption{Conditional Branch Instructions in MIPS}
\begin{center}
\includegraphics[width=\textwidth, height=0.5\textheight]{docs/images/operations-4}
\end{center}
\end{figure}
\end{frame}

\begin{frame}{Operations of the Computer Hardware (Cont'd)}
\begin{figure}\caption{Unconditional Jump Instructions in MIPS}
\begin{center}
\includegraphics[width=\textwidth, height=0.17\textheight]{docs/images/operations-5  }
\end{center}
\end{figure}
\end{frame}

\begin{frame}{Example - Compiling a Complex C Assignment into MIPS}
\begin{flushleft}
A somewhat complex statement contains the five variables \texttt{f}, \texttt{g}, \texttt{h}, \texttt{i}, and \texttt{j}:

\hspace{8mm}\texttt{f = (g + h) – (i + j);}

What might a C compiler produce?
\end{flushleft}
\end{frame}

\begin{frame}[fragile]{Answer}
\begin{itemize}
\item[-]
\texttt{add t0, g, h \# temporary variable t0 contains g + h}

\item[-]
\texttt{add t1, i, j \# temporary variable t1 contains i + j}

\item[-]
\texttt{sub f, t0, t1 \# f gets t0 – t1, which is (g + h) – (i + j)}
\end{itemize}
\end{frame}

\subsection{Operands of the Computer Hardware}
\begin{frame}{Example - Compiling a C Assignment Using Registers}
\begin{flushleft}
It is the compiler’s job to associate program variables with registers. 

Take, for instance, the assignment statement from our earlier example:

\hspace{8mm}\texttt{f = (g + h) – (i + j);}

The variables \texttt{f}, \texttt{g}, \texttt{h}, \texttt{i}, and \texttt{j} are assigned to the registers \texttt{\$s0}, \texttt{\$s1}, \texttt{\$s2}, \texttt{\$s3},
and \texttt{\$s4}, respectively. 

What is the compiled MIPS code?
\end{flushleft}
\end{frame}

\begin{frame}[fragile]{Answer}
\begin{itemize}
\item[-]
\texttt{add \$t0, \$s1, \$s2  \# register \$t0 contains g + h}

\item[-]
\texttt{add \$t1, \$s3, \$s4  \# register \$t1 contains i + j}

\item[-]
\texttt{sub \$s0, \$t0, \$t1  \# f gets \$t0 – \$t1, which is (g + h) – (i + j)}
\end{itemize}
\end{frame}

\subsubsection{Memory Operands}
\begin{frame}{Example - Compiling Using Load and Store}
\begin{flushleft}
Assume variable \texttt{h} is associated with register \texttt{\$s2} and the base address of the array \texttt{A} is in \texttt{\$s3}. 

What is the MIPS assembly code for the C assignment state­
ment below?

\hspace{8mm}\texttt{A[12] = h + A[8];}
\end{flushleft}
\end{frame}

\begin{frame}{Answer}
\begin{itemize}
\item[-]
\texttt{lw \$t0, 32(\$s3) \# Temporary reg \$t0 gets A[8]}

\item[-]
\texttt{add \$t0, \$s2, \$t0  \# Temporary reg \$t0 gets h + A[8]}

\item[-]
\texttt{sw \$t0, 48(\$s3)  \# Stores h + A[8] back into A[12]}
\end{itemize}    
\end{frame}

\subsubsection{Constant or Immedidate Operands}
\begin{frame}{Constant or Immedidate Operands}
\begin{itemize}
\item[-] \texttt{addi \$s3, \$s3, 4  $\, \, \, \, \,$ \# \$s3 = \$s3 + 4}
\end{itemize}
\end{frame}

\subsection{Representing Instructions}
\begin{frame}{Insructions Big Picture}
\begin{figure}
\begin{center}
\includegraphics[width=\textwidth, height=0.55\textheight]{docs/images/instructions}
\end{center}
\end{figure}
\end{frame}
\subsection{Logical Operations}
\subsection{Instuctions for Making Decisions}
\subsection{Suporting Precedures in Computer Hardware}
\subsection{MIPS Addressing for 32-Bit Immediates and Addresses}
\subsection{A C Sort Example to Put It All Together}