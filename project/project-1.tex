\documentclass[11pt, dvipsnames, svgnames, x11names]{article}

% URLs and hyperlinks ---------------------------------------
\usepackage{hyperref}
\hypersetup{
    colorlinks=true,
    linkcolor=blue,
    filecolor=magenta,      
    urlcolor=blue,
}
\usepackage{xurl}
%----------------------------------------------------

\usepackage{xepersian}
\settextfont{Yas}
\setdigitfont{Yas}

\title{پروژه}
\date{}

\begin{document}
\maketitle    
\tableofcontents

\begin{abstract}
در درس معماری، شما معماری یک پردازنده‌ی 
\lr{MIPS}
را بررسی نمودید، و برای یادگیری بهتر آنچه آموختید یک پروژه طراحی پردازنده برای شما طراحی شده است.

پروژه را شما به صورت تیمی انجام خواهید داد. فرقی ندارد که در کدام سکشن از درس ثبت‌نام کرده‌اید، می‌توانید از افراد سکشن دیگر نیز به عنوان هم‌تیمی کسی را انتخاب کنید.

شما باید برای پروژه‌تان یک داکیومنتیش هم بنویسید، این یعنی تنها فایل لاجیسیم و اجرای آن کافی نیست. اما این به این معنی نیست که باید یک داکیومنتیشن بسیار مفصل و جامع تحویل بدهید، بلکه باید برای تصمیم‌هایی که گرفته‌اید دلیل بنویسید.

فاز بندی انجام شده در این سند، صرفا برای نظم دهی به قسمت‌های مختلف پروژه است و ما چندین زمان تحویل \underline{نخواهیم} داشت.

پردازنده‌ای که طراحی می‌کنید را یک پردازنده‌ی \underline{۱۶} بیتی در نظر بگیرید. یعنی اعداد ورودی و خروجی ۱۶ بیتی هستند.

زمان تحویل \textbf{نهایی} پروژه ساعت ۲۳:۵۹:۵۹ ثانیه روز ۱۴ دی ماه ۱۴۰۲ است.
\end{abstract}

\newpage
\section{آماده سازی}
\subsection{تیم سازی}
ابتدا یک تیم متشکل از ۳ یا ۴ نفر تشکیل دهید. سپس سرگروه تیم نام خود و اعضای گروه خود را در \href{https://docs.google.com/spreadsheets/d/1Abn-jLsgmaDNbdNiK79yFFMcXp9WaEA_Rby0RCvrBUw/edit#gid=0}{این برگه}
 بنویسد.

\subsection{نرم‌افزار}
برای انجام این پروژه شما به نرم‌افزار 
\lr{Logisim}
احتیاج دارید.

برای دریافت این نرم‌افزار به \href{http://www.cburch.com/logisim/}{\texttt{http://www.cburch.com/logisim/}}
مراجعه کنید.

\subsection{گرفتن دستورات مخصوص تیم}
بعد از تیم‌سازی، سرگروه تیم‌ها لطفا در تلگرام به من
\lr{(@pyeafp)}
پیام دهند تا \lr{Instruction}های مخصوص تیمشان را دریافت کنند.

\section{فاز اول،‌ طراحی}\label{design}
در اولین فاز این پروژه، شما باید بر اساس دستوراتی که دریافت کرده‌اید، موارد زیر را انجام دهید.
\begin{itemize}
\item 
طراحی \lr{ALU}\RTLfootnote{برای راهنمایی می‌توانید فصل پنجم از \href{https://github.com/mahdihaghverdi/cpu/blob/main/docs/documentation.pdf}{این سند} را بررسی کنید}،
\item 
بخش‌بندی \lr{ISA}ی داده شده\RTLfootnote{برای راهنمایی می‌توانید فصل دوم \href{https://github.com/mahdihaghverdi/cpu/blob/main/docs/documentation.pdf}{این سند} را بررسی کنید}،
\item 
کدگذاری \lr{ISA}ی خود\RTLfootnote{برای راهنمایی می‌توانید فصل دوم \href{https://github.com/mahdihaghverdi/cpu/blob/main/docs/documentation.pdf}{این سند} را بررسی کنید} و
\item 
طراحی سیگنال‌های کنترلی لازم\RTLfootnote{برای راهنمایی می‌توانید فصل چهارم \href{https://github.com/mahdihaghverdi/cpu/blob/main/docs/documentation.pdf}{این سند} را بررسی کنید}
\end{itemize}

نکات:
\begin{enumerate}
\item 
تمامی موارد گفته شده، روی کاغذ هستند و باید در سند پروژه نوشته شوند.

برای مثال در دستوراتی که به تیم 
\lr{x}
داده شده، تنها دو دستور محاسباتی
\lr{add}
و
\lr{addi}
وجود دارد، این به این معنی‌ است که 
\lr{ALU}
این تیم \underline{حداقل} باید بتواند جمع انجام بدهد و شاید هیچ نیازی به پشتیبانی از عمل‌های اصلی دیگر را نداشته باشد، به این موضوع دقت کنید.
\item 
در زمان ارائه برای تصمیم‌هایی که گرفتید، باید بتوانید دلیل ارائه دهید.
\end{enumerate}

\section{فاز دوم، اسمبلر}
برای 
\lr{ISA}ی
خود و کدگذاری خودتان، یک 
\lr{Assembler}
با زبان برنامه نویسی دلخواه بنویسید.\RTLfootnote{برای راهنمایی می‌توانید \href{https://github.com/mahdihaghverdi/cpu/tree/main/pyasm}{این اسمبلر}  را بررسی کنید}

\section{فاز سوم، لاجیسیم}
بر اساس طراحی انجام شده در فاز اول، شما باید پردازنده‌ی خود را با نرم‌افزار لاجیسیم طراحی کنید.

\section{ارائه}
\subsection{روند ارائه}
\begin{enumerate}
\item
در روز ارائه از طراحی‌های انجام شده در قسمت 
\ref{design}
از شما پرسش‌هایی خواهد شد.

\item 
سپس از شما خواسته خواهد شد که یک یا دو تا از دستوراتی که به شما داده شده است را به صورت کد اسمبلی نوشته و آن را با اسمبلر به صفر و یک تبدیل کنید

\item 
سپس باید \lr{instruction} تولید شده را، به طراحی خود بدهید و خروجی را بررسی کنید

برای مثال در دستورات داده شده به شما، دستور 
\lr{\texttt{add}}
وجود دارد، به شما گفته می‌شود که جمع 
$x = 2 + 3$
را به اسمبلی نوشته، آنرا با اسمبلر به صفر و یک تبدیل کرده و اجرا کنید و خروجی ۵ را نمایش دهید.
\end{enumerate}
\end{document}