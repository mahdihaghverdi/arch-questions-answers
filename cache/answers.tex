\documentclass[11pt, dvipsnames, svgnames, x11names]{article}

% URLs and hyperlinks ---------------------------------------
\usepackage{hyperref}
\hypersetup{
    colorlinks=true,
    linkcolor=blue,
    filecolor=magenta,      
    urlcolor=black,
}
\usepackage{xurl}
%----------------------------------------------------

\usepackage{amsmath, amssymb}

% footnotes in headings -----------------------------
\usepackage[stable]{footmisc}
%----------------------------------------------------

\usepackage{adjustbox}
\usepackage{float}
\usepackage{listings}
\usepackage{color}
\usepackage{xcolor}
\usepackage{makecell}

\definecolor{dkgreen}{rgb}{0,0.6,0}
\definecolor{gray}{rgb}{0.5,0.5,0.5}
\definecolor{mauve}{rgb}{0.58,0,0.82}

\lstset{frame=tb,
    language=vhdl,
    aboveskip=3mm,
    belowskip=3mm,
    showstringspaces=false,
    columns=flexible,
    basicstyle=\ttfamily,
    numbers=left,
    numberstyle=\small\color{gray},
    keywordstyle=\bfseries\color{Green4},
    commentstyle=\color{gray},
    stringstyle=\color{mauve},
    breaklines=true,
    breakatwhitespace=true,
    tabsize=4,
    identifierstyle=\color{black}
}

\usepackage{xepersian}
\settextfont{Yas}
\setdigitfont{Yas}

\title{پاسخ تمرین فصل پنجم - کَش}
\date{}

\begin{document}
\maketitle    
\tableofcontents

\section{\lr{Miss Penalty}}
\begin{flushleft}
\lr{Miss Penalty = \texttt{2 + 2 × 20 + 2 × 2 = 46} memory bus cycles}
\end{flushleft}

\section{محاسبات}
1. 
\begin{latin}
\begin{equation}
\text{CPI} = \text{CPIbase + Mem Stalls Per Instruction}
\end{equation}
\begin{equation}
\begin{split}
\text{\texttt{Mem Stalls/Instruction}} = 
& \ \text{\texttt{ICache miss rate}} \\
\times & \ \text{\texttt{ICache miss penalty}} \\
+ & \ \text{\texttt{DCache miss rate}} \\
\times & \ \text{\texttt{DCache miss penalty}} \\
\times & \ \text{\texttt{(Data Mem access/instruction}} \\
= & \ \frac{0.6}{100} \times 150 + \frac{35}{100} \times \frac{7}{100}  \times 150 \\
= & \ 0.9 + 3.675 = 4.575 \\
\end{split}
\end{equation}

\begin{equation}
\text{CPI} = 1.2 + 4.575 = 5.775
\end{equation}
\end{latin}

2. 
پردازنده ایده آل
\lr{$\frac{5.775}{1.2} = 4.8125$}
برابر سریعتر از پردازنده با \lr{cache} است.

\vspace{5mm}
3. 
\begin{latin}
\begin{equation}
\begin{split}
\text{CPI}_{\text{no-cache}} = & \ 1.2 \\
+ & \ [1 \, \text{(for instruction fetch)} 
+ \ 0.35 \, \text{(for data access)}] \\ 
\times & \ 150 = 203.7 
\end{split}
\end{equation}
\end{latin}

\section{\lr{Associative cache}}

اندازه یک
\lr{set}:
\begin{latin}
\[4 \ \text{\lr{(bytes per word)}} \times 16 \ \text{\lr{(words per block or way)}} \times 4 \ \text{\lr{(ways)}} = 2^8\]
\end{latin}

اندازه کل داده در این \lr{cache} بر حسب بایت:
\lr{$8 \ \text{\lr{KByte}} = 2^{13}$}
است پس این 
\lr{cache}
کلا
\begin{latin}
\[\frac{2^{13}}{2^8} = 2^5 = 32\]
\end{latin}
 عدد 
\lr{set}
دارد.

پس تعداد بیتهای \lr{tag} معادل خواهد بود با
\begin{latin}
\[32 - (2 + 4 + 5) = 21\]
\end{latin}
(2 بیت برای آفست بایت و 4 بیت برای آفست کلمه و 5 بیت برای \lr{index})

\vspace{5mm}
اندازه کل یک block معادل است با:
\begin{latin}
\[1 \ \text{\lr{bit (valid)}} + 21 \ \text{\lr{bits (tag)}} + 16 \times 4 \ \text{\lr{bytes (Data)}} = 534 \ \text{\lr{bit}} = 66.75  \ \text{\lr{Byte}}\]
\end{latin}

اندازه کل \lr{cache} مساوی است با:
\begin{latin}
\[66.75 \times 4 \ \text{\lr{(set/block)}} \times 32 \ \text{\lr{(sets/cache)}} = 8544 \ \text{\lr{Bytes}}\]
\end{latin}
\section{\lr{AMAT}}

\begin{latin}
\begin{equation}
\text{AMAT} = \text{Hit Time}_{\text{L1}} +\text{Miss Rate}_{\text{L1}} \times \text{Miss Penalty}_{\text{L1}}
\end{equation}
\begin{equation}
\begin{split}
\text{Miss Penalty}_{\text{L1}} = & \ \text{Hit Time}_{\text{L2}} \\
+ & \ \text{Miss Rate}_{\text{L2}} \\
\times & \ \text{Miss Penalty}_{\text{L2}} \\
= & 18 + 20 \% \times 500 = 118 \ \text{cycles}
\end{split}
\end{equation}
\begin{equation}
\text{AMAT} = 4 + 5\% \times 118 = 9.9 \ \text{cycles}
\end{equation}
\end{latin}
\end{document}