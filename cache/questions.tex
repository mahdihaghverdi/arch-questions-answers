\documentclass[11pt, dvipsnames, svgnames, x11names]{article}

% URLs and hyperlinks ---------------------------------------
\usepackage{hyperref}
\hypersetup{
    colorlinks=true,
    linkcolor=blue,
    filecolor=magenta,      
    urlcolor=black,
}
\usepackage{xurl}
%----------------------------------------------------

% footnotes in headings -----------------------------
\usepackage[stable]{footmisc}
%----------------------------------------------------

\usepackage{adjustbox}
\usepackage{float}
\usepackage{listings}
\usepackage{color}
\usepackage{xcolor}
\usepackage{makecell}

\definecolor{dkgreen}{rgb}{0,0.6,0}
\definecolor{gray}{rgb}{0.5,0.5,0.5}
\definecolor{mauve}{rgb}{0.58,0,0.82}

\lstset{frame=tb,
    language=vhdl,
    aboveskip=3mm,
    belowskip=3mm,
    showstringspaces=false,
    columns=flexible,
    basicstyle=\ttfamily,
    numbers=left,
    numberstyle=\small\color{gray},
    keywordstyle=\bfseries\color{Green4},
    commentstyle=\color{gray},
    stringstyle=\color{mauve},
    breaklines=true,
    breakatwhitespace=true,
    tabsize=4,
    identifierstyle=\color{black}
}

\usepackage{xepersian}
\settextfont{Yas}
\setdigitfont{Yas}

\title{تمرین فصل پنجم - کَش}
\date{}

\begin{document}
\maketitle    
\begin{abstract}        
سوالات فصل پنجم کتاب، که در مورد حافظه‌ی نهان و سیاست‌های آن صحبت می‌کند، برای شما آماده شده‌اند.

پاسخ هر سوال را در قسمت مربوط آنها در کوئرا به صورت \lr{PDF} به صورت تایپ‌ شده، یا دست‌نویس خوش‌خط و خوانا آپلود کنید.               

پس از پایان‌ یافتن زمان ارسال تمرین، پاسخ‌های این تمرین در آدرس زیر قرار خواهد گرفت.
\begin{flushleft}
\url{https://github.com/mahdihaghverdi/arch-questions-answers/tree/main/cache}
\end{flushleft}

\end{abstract}
\tableofcontents

\section{\lr{Miss Penalty}}
یک
\lr{cache}
با بلوک‌های 4 کلمه‌ای
\lr{(4 word cache block)}
را در نظر بگیرید.
ساختار حافظه
\lr{(DRAM)}
و باس
\lr{(Bus)}
داده آن بصورت 2 کلمه‌ای است (در هر آدرس، 2 کلمه قرار دارد). فرض کنید 2 سیکل برای ارسال آدرس به \lr{DRAM} نیاز است. همچنین هر دسترسی به 
\lr{DRAM}
به 20 سیکل زمان نیاز داشته و ارسال هر داده بر روی باس به 2 سیکل زمان احتیاج دارد. برای این سیستم حافظه، 
\lr{Miss Penalty}
را محاسبه کنید.

\section{محاسبات}

یک پردازنده دارای دو 
\lr{L1 cache}
مجزا برای \lr{instruction} و \lr{data} است. مشخصات این سیستم در زیر آمده است.

\begin{latin}
\begin{itemize}
\item CPIbase= 1.2  (Ideal Situation)
\item Main memory access time=150 cycles
\item Instruction mix: 47\% arithmetic-logic, 35\% load-store, 18\% control
\item I-Cache miss = 0.6\%
\item D-Cache miss = 7\%
\end{itemize}
\end{latin}

\begin{enumerate}
\item 
\lr{CPI} 
کل این سیستم را با فرض وجود \lr{cache} محاسبه کنید.

\item 
پردازنده ایدآل (بدون \lr{stall}) چقدر از پردازنده دارای \lr{cache} سریعتر است؟ (پاسخ را بصورت یک کسر میتوانید نمایش دهید)

\item 
CPI 
کل این سیستم را با فرض عدم وجود \lr{cache} محاسبه کنید.
\end{enumerate}

\section{\lr{Associative cache}}
یک
\lr{4-way set associative cache}
را با
\begin{itemize}
\item \lr{4-byte word size}،
\item \lr{16-word block size} و
\item \lr{8-KB total data cache size}
\end{itemize}
را در یک پردازنده با آدرس‌دهی 32 بیتی در نظر بگیرید.

اندازه کل این \lr{cache} را بر حسب بایت (یا \lr{KB}) محاسبه کنید. (فقط پاسخ صحیح نمره دارد) 

\section{\lr{AMAT}}
مقدار \lr{AMAT} را بر حسب تعداد سیکل، برای سیستم حافظه زیر محاسبه کنید. 
\begin{latin}
\begin{table}[H]
\begin{adjustbox}{width=\textwidth}
\begin{tabular}{|l|}
\hline
L1 2-way set associative cache with
hit ratio=95\% and access time=4 cycles\\
\hline
L2 direct mapped cache with
hit ratio=80\% and  access time=18 cycles \\
\hline
Main memory access time=500 cycles \\
\hline
\end{tabular}
\end{adjustbox}
\end{table}
\end{latin}
\end{document}